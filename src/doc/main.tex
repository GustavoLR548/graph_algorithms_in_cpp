\documentclass[12pt]{article}

\usepackage{sbc-template}

\usepackage{graphicx,url}

%\usepackage[brazil]{babel}   
\usepackage[utf8]{inputenc}  

     
\sloppy

\title{Implementação de grafos usando C++}

\author{Gustavo Lopes Rodrigues\inst{1}, Thiago Henriques Nogueira\inst{2},}


\address{Instituto de Ciências e Informática \\Pontifícia Universidade Católica de Minas Gerais
  (PUC-MG)\\}


\begin{document} 

\maketitle

\begin{abstract} 
  This document serves as a guide to explain how it
  was the implementation
  of 4 different types of graphs using the C ++ language. The types of
  graphs are: a) Unweighted directed graph, b) Unweighted non-directed graph
  , c) Weighted targeted graph, d) Weighted non-directed graph.
\end{abstract}

\begin{resumo} 
  Este documento serve como guia para explicar como 
  foi a implementação
  de 4 tipos diferentes de grafos usando a linguagem C++. Os tipos de 
  grafos são: a) Grafo direcionado não-ponderado,b) Grafo não-direcionado 
  não-ponderado,c) Grafo direcionado ponderado,d) Grafo não-direcionado 
  ponderado.
\end{resumo}


\section{Informações gerais}

  Este é um trabalho feita para a disciplina de Algoritmo em Grafos, para a \emph{Pontifícia Universidade Católica de Minas Gerais},
  onde foi necessário a construção de 4 tipos de grafos diferentes em C++.

  \begin{itemize}
    \item Grafo direcionado não-ponderado
    \item Grafo não direcionado não ponderado 
    \item Grafo direcionado ponderado
    \item Grafo não-direcionado ponderado
  \end{itemize}

  Para a interação com tais grafos, foi feito um sistema de menu, onde o usuário pode interagir com o programa,
  logo, escolher quais operações gostaria de fazer, além de escolher em qual grafo.

  \section{MenuOptions.cpp}

  A classe MenuOptions.cpp foi elaborada com o propósito de organizar todos os menus assim como a tomada de decisões do programa.

  Primeiro, o objeto irá resgatar o menu a partir de um arquivo de texto, e armazenar
  tudo dentro de um vetor com vetores de texto(std::string). O index do menu começara
  sempre no início do vetor, e assim que ele for necessário, o usuário pode imprimir na tela, usando:
  \emph{printcurrentmenu}.

  Além disso o programa pode interpretar entrada do usuário, com a função
  \emph{intepretuserinput}, onde o programa irá verificar se é uma entrada
  válida, além de retornar alguma operação(Se houver a necessidade de realizar)
  alguma operação.

  \section{Matriz}

  A class Matriz.cpp foi criada para a utilização de busca por matriz no
  grafo. Em alguns dos casos, fazer busca pela Matriz pode se mostrar mais 
  interessante em uma matriz do que usando um vector.

  Além disso, a matriz foi criada usando um sistema de template(T), em outras
  palavras, esta classe possui uma abstração em relação ao conteúdo inserida
  na mesma, o usuário pode tanto inserir um tipo de dado primitivo(int, char) ou até
  mesmo outras classes.

  Para a utilização da matriz, apenas é preciso enviar para o classe o 
  tamanho da mesma, e então o programa irá criar dinamicamente as células.

  Eis aqui algumas das funcionalidades que a matriz possui.

  \subsection{get}

  Para pegar o conteúdo T em alguma posição específica da matriz, se usa 
  a função get, que recebe a posição da célula em forma de sistema cartesiano
  (x,y), e retorna o elemento contido na posição.

  \subsection{insert} 

  Assim como é possivel resgatar um elemento T da matriz, é possível inserir
  um também, usando a função insert. Da mesma forma que a função get, esta 
  recebe a posição da célula no eixo das coordenadas e abscissas, e retorna
  um booleano, se foi possível inserir. 

  \subsection{Expandir a matriz}

  Uma das funções especiais dessa matriz em específica, é que ela pode
  expandir de acordo com a necessidade dos usuários. Vamos supor que o mesmo
  começe com uma matrix 2x2 e queira inserir mais elementos dentro dela. Neste
  caso, ele pode inserir esse elemento, expandido a matriz, seja de forma quadrada,
  vertical ou horizontal.

\end{document}